ON ANOTHER'S SORROW
   Can I see another's woe,
   And not be in sorrow too?
   Can I see another's grief,
   And not seek for kind relief?

   Can I see a falling tear,
   And not feel my sorrow's share?
   Can a father see his child
   Weep, nor be with sorrow filled?

   Can a mother sit and hear
   An infant groan, an infant fear?
   No, no! never can it be!
   Never, never can it be!
   
   And can He who smiles on all
   Hear the wren with sorrows small,
   Hear the small bird's grief and care,
   Hear the woes that infants bear—
   
   And not sit beside the next,
   Pouring pity in their breast,
   And not sit the cradle near,
   Weeping tear on infant's tear?
   
   And not sit both night and day,
   Wiping all our tears away?
   Oh no! never can it be!
   Never, never can it be!
   
   He doth give his joy to all:
   He becomes an infant small,
   He becomes a man of woe,
   He doth feel the sorrow too.
   
   Think not thou canst sigh a sigh,
   And thy Maker is not by:
   Think not thou canst weep a tear,
   And thy Maker is not near.
   
   Oh He gives to us his joy,
   That our grief He may destroy:
   Till our grief is fled an gone
   He doth sit by us and moan.