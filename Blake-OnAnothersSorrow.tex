\poemtitle{On Another's Sorrow}

\begin{poem}

\begin{stanza}
Can I see another's woe,\verseline
And not be in sorrow too?\verseline
Can I see another's grief,\verseline
And not seek for kind relief?
\end{stanza}
\begin{stanza}
\verseindent Can I see a falling tear,\verseline
And not feel my sorrow's share?\verseline
Can a father see his child\verseline
Weep, nor be with sorrow filled?
\end{stanza}
\begin{stanza}
\verseindent Can a mother sit and hear\verseline
An infant groan, an infant fear?\verseline
No, no! never can it be!\verseline
Never, never can it be!
\end{stanza}
\begin{stanza}
\verseindent And can He who smiles on all\verseline
Hear the wren with sorrows small,\verseline
Hear the small bird's grief and care,\verseline
Hear the woes that infants bear—
\end{stanza}
\begin{stanza}
\verseindent And not sit beside the next,\verseline
Pouring pity in their breast,\verseline
And not sit the cradle near,\verseline
Weeping tear on infant's tear?
\end{stanza}
\begin{stanza}
\verseindent And not sit both night and day,\verseline
Wiping all our tears away?\verseline
Oh no! never can it be!\verseline
Never, never can it be!
\end{stanza}
\begin{stanza}
\verseindent He doth give his joy to all:\verseline
He becomes an infant small,\verseline
He becomes a man of woe,\verseline
He doth feel the sorrow too.
\end{stanza}
\begin{stanza}
\verseindent Think not thou canst sigh a sigh,\verseline
And thy Maker is not by:\verseline
Think not thou canst weep a tear,\verseline
And thy Maker is not near.
\end{stanza}
\begin{stanza}
\verseindent Oh He gives to us his joy,\verseline
That our grief He may destroy:\verseline
Till our grief is fled an gone\verseline
He doth sit by us and moan.
\end{stanza}

\end{poem}